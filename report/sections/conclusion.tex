\section{Conclusion}

We found that linguistic information can help boost overall classification
accuracy by a small amount, especially in the case where the amount of available
training data is small.
Additionally, we observed that the models trained on soft labels achieved their
maximum validation accuracy in fewer epochs than those trained on 1-hot labels.
We also confirmed our hypothesis that the neural networks trained on soft
labels would make more semantically reasonable errors than those trained the
traditional way.


\subsection{Future Work}

We believe that this work shows promise for using soft labels, but we have only
begun to touch the surface.
First, we would like to run many more experiments with different
hyperparameters and affinity matrix normalizations, and take the average
performance scores of many repeated iterations of the same experiment.  We can
also explore different Word2Vec training sets and many other WordNet distance
metrics, and compare how each does on a variety of data sets.

Next, we would like to expand the size and scope of our training data, as we
have only several subsets of 25 ImageNet classes. Ultimately, we would like to
train on the entirety of ImageNet, and use data augmentation methods to come
closer to current state-of-the-art accuracy scores.
Another natural extension is to train a wider variety of deep neural network
architectures, including simpler models (with a much smaller number of
parameters) to test our hypothesis that we can train simpler models faster,
as shown in \cite{hinton2015distilling}.

Finally, we would like to see how well our method extends to pre-trained models
(fine-tuning) and whether we can get any reasonable performance in zero-shot
learning. Zero-shot learning could potentially be achieved by training the CNN
on images of classes that are semantically similar to a class which we have no
training examples for. By emphasizing that certain features contribute to the
unseen object category through soft labels, we believe that some information
necessary to identify the unknown class would be learned by the model.
