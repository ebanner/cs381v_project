\section{Conclusion}

We found that by training with soft labels can help boost overall classification
accuracy by a small amount, especially in the case where the amount of available
training data is small. We were able to do this \emph{cheaply} with off the
shelf linguistic resources and simple image features. We also confirmed
our hypothesis that the neural networks trained on soft labels would make more
semantically reasonable errors than those trained the traditional way.


\subsection{Future Work}

We believe that this work shows promise for using soft labels, but we have only
begun to scratch the surface. First, we would like to run many more experiments
with different hyperparameters and affinity matrix normalizations, and take the
average performance scores of many repeated iterations of the same experiment.
We can also explore different corpora for learning word embeddings and many
other WordNet distance metrics, and compare how each does on a variety of data
sets. Additionally, we observed that the models trained on soft labels achieved
their maximum validation accuracy in fewer epochs than those trained on 1-hot
labels and would like to run more experiments to confirm this finding.  

Next, given that we saw more improvement on C10 than C5, we would like to expand
the size and scope of our training data. Ultimately, we would like to train on
the entirety of ImageNet, and use data augmentation methods to come closer to
current state-of-the-art accuracy scores. Another natural extension is to train
a wider variety of deep neural network architectures, including simpler models
(with a much smaller number of parameters) to test our hypothesis that we can
train simpler models faster, as shown in \cite{hinton2015distilling}.

We are also interested in generating soft labels at the \emph{image} level,
rather than the \emph{class} level. In this setting, the class-level semantic
similarities could be used as a weak prior, which could be refined by visual
features. One could go one step up and use a weak classifier to inform these
soft labels. Such an approach would go one step further in uniting the
distillation work in \cite{hinton2015distilling} and work in more traditional
semantic label sharing settings.

Finally, we would like to see how well our method extends to pre-trained models
(fine-tuning) and whether we can get any reasonable performance in zero-shot
learning. Zero-shot learning could potentially be achieved by training the CNN
on images of classes that are semantically similar to a class which we have no
training examples for. By emphasizing that certain features contribute to the
unseen object category through soft labels, we believe that some information
necessary to identify the unknown class would be learned by the model.
