\section{Related Work}


Incorporating semantic information into object classification has been well
studied. Many prior works have explored the relationships between language and
vision, and have attempted to use these results to improve classification
performance \cite{izadinia2015segment, frome2013devise}.
By exploiting the the lingustic semantic relationships between object classes
in WordNet \cite{miller1995wordnet}, \cite{marszalek2007semantic}, and
\cite{grauman2011learning} perform hierarchical classification by
discriminating images at more abstract categories first, and then refining the
classification by moving down the WordNet tree.
It has also been shown that exploiting visual similarity between general object
categories can help a hierarchical classifier learn to discriminate those image
classes more effectively \cite{li2010building}.
Moreover, \cite{deselaers2011visual} have shown that visual similarity and
linguistic semantics in WordNet are correlated. This justifies using language
as a means of improving visual feature learning in convolutional neural
networks.


The authors of DeViSE \cite{frome2013devise} train a deep neural network to
learn a mapping between visual embeddings and word embeddings trained with
Word2Vec \cite{mikolov2013distributed}. In doing so, their model learns the
semantic relationships between image categories. While their performance does
not improve over existing state-of-the-art classifiers, they show that their
errors are more semantically reasonable; that is, on average the misclassified
categories are more closely related to the ground truth in Word2Vec feature
space.


In \cite{hinton2015distilling}, the authors introduce a method called
\emph{distillation}. The idea is that the probability distribution of a neural
network's softmax layer reflects the way the model generalizes different
classes (e.g. a ``BMW'' is likely to be closer to a ``garbage truck'' than to a
``carrot'' in the model's learned feature space). A simpler and more
computationally efficient model can then be trained on the soft labels produced
by this more advanced model.
The goal of this method is to provide the end user with a fast and lightweight
model for doing real-time classification instead of needing to rely on complex
and often slow deep neural network architectures.
Unlike the work in this paper, our intent is to leverage linguistic information
to obtain the soft labels, inherently providing the model more information to
learn from. This can significantly reduce the amount of required trianing data,
as shown in the results of \cite{hinton2015distilling}. Insufficient data would
typically result in severe overfitting when using classic 1-hot labels.


\cite{zhao2011large} have proposed an idea that is most closely related to our
own work.
They used the lingustic semantics information in WordNet to create soft labels
on ImageNet categories. In particular, they defined the semantic distance
between WordNet categories as
$$D_{ij} = \frac{\mathrm{intersect(path}(i), \mathrm{path}(j))}{\mathrm{max(length(path}(i)), \mathrm{length(path}(j)))}$$
where $\mathrm{path}(i)$ is the path from the root node to node $i$ and
$\mathrm{intersect}(p_1, p_2)$ is the number of nodes that are shared by both
paths. Given this distance metric, they constructed a semantic relatedness
matrix $\mathbf{S}_{ij} = \exp(-\kappa(1-D_{ij}))$ where $\kappa$ is a
parameter that controls the semantic relatedness decay factor.
They then trained a multi-way classifier on these modified label vectors. They
were able to achieve higher performance with the soft labels on this large data
set using a softmax classifier on SIFT visual words.


Contrary to the approach of \cite{zhao2011large}, we train an end-to-end
convolutional neural network which learns the features directly.
We believe that this is more effective because the soft labels can influence
the properties of the convolution filters at all layers of the neural network.
This takes advantage of the hierarchical nature of CNNs where the initial
layers tend to handle picking out low-level features and later layers learn to
discriminate more visually semantic attributes. Soft labels can guide the
pipeline at all stages of this convolutional hierarchy.
%We believe that this is more effective because the soft labels can influence
%the properties of the image filters, thereby modeling the entire pipeline to
%take advantage of visual similarity factors at various semantic levels of the
%images. This reflects the hierarchical visual semantics often discovered by the
%different layers of the neural network \cite{?}.
We also expand on this work by studying several soft labeling schemes in
addition to WordNet. Specifically, we compare and analyze classification
results using Word2Vec semantics and various visual similarity schemes.
